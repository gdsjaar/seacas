\newif\ifdraft\draftfalse
\newif\ifsand\sandtrue

\SANDprintDate{\today}
\SANDnum{SAND88-0616}
\SANDauthor{G. D. Sjaardema and G. W. Wellman\\
        Applied Mechanics Division I}
\newcommand{\theTitle}{Numerical and Analytical Methods for Approximating the 
Eccentric Impact Response (Slapdown) of Deformable Bodies}

\title{\theTitle}
\ifsand
\pdfbookmark[1]{Cover}{cover}
\doCover
\newpage
\else
\SANDmarks{cover}
\setcounter{page}{3}
\fi

\pdfbookmark[1]{Title}{title}

%\begin{titlepage}
\begin{center}
\SANDnumVar\\
\SANDreleaseTypeVar\\
\ifdraft
Draft Date: \SANDprintDateVar\\
\else
Printed \SANDprintDateVar\\
\fi

\vspace{0.75in}
\CoverFont{m}{24}{28pt}
\theTitle\\
\vspace{0.5in}
\CoverFont{m}{12}{14pt}
\SANDauthorVar\\
Simulation Modeling Sciences Department\\
Sandia National Laboratories\\
Albuquerque, NM 87185-0380\\
\vspace*{.4in}
\textbf{Abstract}
\end{center}
\vspace{-.4cm}\par
Analytical and numerical methods have been developed to approximate
the eccentric impact response of deformable bodies.  The eccentric
impact response is commonly known as {\em slapdown} since the
off-center impact gives the body a rotational velocity which causes
impact at the opposite end.  A code, \SLAP , has been written to
approximate the slapdown behavior of deformable bodies. The body is
idealized as a three degree-of-freedom system with nonlinear impact
behavior. 

Several parameters of interest to the analysis and design of laydown
weapons were studied to determine their effects on the secondary
impact velocities (slapdown).  Parameters studied are aspect ratio of
the body, stiffness of the initial impact, and friction between the
target and the body. Rules for applying the results of scale model
tests to full scale bodies are developed and confirmed for nonlinear
spring behavior. 
